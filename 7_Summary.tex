% -*- root: Document.tex -*-

\chapter{Summary}
\label{chap:summ}

In this deliverable, we have presented the initial version of the specification, design and implementation of the \GP{} scheduler.
This scheduler is seamlessly working on top of \textsc{Docker Swarm}, and is so far fully compliant with it.

Efficient VM or container scheduling is particularly critical in cloud data centers to not only provide good performance, but also minimize the hardware resource required for running concurrent applications.
This can, in turn, reduce the costs of operating a cloud infrastructure and, importantly, reduce the associated energy footprint.
In particular, when efficiently packing containers on physical hosts, one can save significant amounts of energy by turning off unused servers.

\GP{} is a new scheduler for containers that borrows ideas from generational garbage collectors.
An original feature of \GP is that it does \emph{not} assume the properties of the containers and workloads to be known in advance.
It relies instead on runtime monitoring to observe the resource usage of containers while in the ``nursery''.
Containers are then run in a young generation of servers, which hold short-running jobs and experience relatively high turnaround.
This collection of servers can also be elastically expanded or shrunk to quickly adapt to the demand.
Long-running jobs are migrated to the old generation, which is composed of more stable and energy-efficient servers.
The containers in the old generation run for a long time and typically experience relatively even load, hence they can be packed in a very efficient way on the servers without need for frequent migrations.

We have implemented \GP in the context of \textsc{Docker Swarm} and evaluated it using a real-world trace.
Our comparison against \textsc{Swarm}'s built-in schedulers shows that \GP does not add noticeable overheads while providing more efficient container packing, which can result in important energy savings.

Our perspectives for \GP includes a careful sensitivity analysis of key parameters like the $k$-means value or the scheduling period.
We also plan to evaluate the performances of \GP in a long-running deployment evolving not only CPU- and memory- intensive containers, but also network- and disk-intensive ones.
