% -*- root: Document.tex -*-

\chapter{User Manual}
\label{chap:usermanual}

\GP{} aims to work seamlessly on top of \textsc{Docker} through a manager of a legacy standalone \textsc{Docker Swarm} cluster.
That means that you can refer to the documentation of \textsc{Docker}\footnote{\textsc{Docker} CLI user guide: \url{https://docs.docker.com/engine/reference/commandline/cli/}.} and \textsc{Docker Swarm}\footnote{Legacy standalone \textsc{Docker Swarm} documentation: \url{https://docs.docker.com/swarm/overview/}.} for any details to use the \textsc{Docker} CLI for deploying containers on a cluster.

First of all, a \textsc{Docker Swarm} cluster have to be deployed (cf. Section~\ref{sec:clustersetup}), and \GP{} have to be run (cf. Section~\ref{sec:genpacksetup}).

The only requirement to use \GP{} when you are deploying some containers is to deploy them first on the generation \texttt{nursery}.
For that, a constraint filter\footnote{\textsc{Docker Swarm} constraint filters documentation: \url{https://docs.docker.com/swarm/scheduler/filter/\#use-a-constraint-filter}.}, for targeting the wanted generation, has to be given in the command of container creation, \textit{e.g.} (assuming that the \textsc{Docker Swarm} manager can be reached on the address \texttt{tcp://0.0.0.0:2380}):

\begin{lstlisting}[language=bash, basicstyle=\small]
$ docker -H :2380 run --name nginx -e constraint:generation==nursery -d nginx
\end{lstlisting}

Then the container \texttt{nginx} is deployed on a node of the generation \texttt{nursery}.
It will be migrated accross the different generations by \GP{}'s generation cycles.


% -*- root: Document.tex -*-

\section{Infrastructure}
\label{sec:infrastructure}

\GP{} is running on top of a \textsc{Docker Swarm} cluster.
To build this one, you need a set of hosts reachable by SSH.
Each host has to have a \textsc{Docker} daemon running and binded on the address \texttt{tcp://0.0.0.0:2375}.


\newpage

% -*- root: Document.tex -*-

\section{Cluster Setup}
\label{sec:clustersetup}

A tiny CLI tool has been implemented to setup easily a \textsc{Docker Swarm} cluster.
This one can be downloaded at \remoteswarmmanagerrepo{}.
Once the configuration for the different hosts used to build the cluster is given (cf. Subsection~\ref{subsec:clustersetup:config}), the CLI can be used to build a \textsc{Docker Swarm} cluster (cf. Subsection~\ref{subsec:clustersetup:manage}).
Note that the example of configuration given here is present in the file \texttt{config.yml.example}: you can copy this file to \texttt{config.yml} and modify it with your own configuration.

When the CLI is called without any command, it returns the list of the available ones:

\begin{lstlisting}[basicstyle=\small]
$ ./remote_swarm_manager.rb
Welcome to the remote Swarm manager

Version: 0.1.0
Use one or several (you can chain them) of the following commands:

- help		Print this usage notice
- version	Print used version of Swarm
- login		Print command to login on each cluster node (manually)
- pull		Pull image given in arg, Swarm image by default
- create	Create a Swarm cluster
- create-nodes	Create Docker nodes
- remove	Remove the current cluster
- ps		Print containers on swarm docker manager
- info		Print infos about swarm docker manager
- monitor	Run cAdvisor on cluster nodes for monitoring
- rm-monitor	Remove cAdvisor and PowerAPI on cluster nodes
- hostnames	Set hostname for each node
- export-key	Export your public key on each VM: export-key [filepath]
- swarm-version	Print used version of Swarm
- ping		Check if connexions for each VM are well configured

For example:	 ./remote_swarm_manager.rb remove monitor create run
\end{lstlisting}

% -*- root: Document.tex -*-

\subsection{Requirements and dependencies}
\label{subsec:clustersetup:requirements}

The CLI requires \textsc{Ruby} v. \texttt{2.3.1}.
If you don't have this version of \textsc{Ruby} installed on your system, you can use \textsc{RVM}\footnote{Install \textsc{RVM}, Ruby Version Manager: \url{https://rvm.io/}.} to easily install it.

The CLI uses some dependencies defined in the file \texttt{Gemfile}. You need to use \textsc{Bundler}\footnote{\textsc{Bundler}, package manager for \textsc{Ruby}: \url{http://bundler.io/}.} to install them.
It can be installed with the command:

\begin{lstlisting}[language=bash, basicstyle=\small]
$ gem install bundler
\end{lstlisting}

Using \textsc{Bundler}, you can install all \textsc{Ruby} dependencies with the command:

\begin{lstlisting}[language=bash, basicstyle=\small]
$ bundle install
\end{lstlisting}

Some of the \textsc{Ruby} dependencies have to be built as native extensions.
For that, \textsc{Bundler} will use \textsc{Make} and \textsc{GCC}: be sure to have those installed on your systems before running the command above.


% -*- root: Document.tex -*-

\subsection{Configuration}
\label{subsec:clustersetup:config}

The configuration has to be given into the YAML file \texttt{config.yml} at the root of tool's code.
This file has to be build like in the example~\ref{remote-swarm-manager-config}.
All configurations options are detailed below.

\begin{minipage}{\linewidth} %avoid splitting
% \vspace{10pt}
\begin{lstlisting}[language=YAML,caption={Configuration file example for \textsc{RemoteSwarmManager} CLI.},label=remote-swarm-manager-config][t]
ssh:
  user: 'host_user'
  identity_file: 'path/to/ssh_keys/id_rsa'
  public_key_file: 'path/to/ssh_keys/id_rsa.pub'
  proxy:
    user: 'proxy_user'
    host: 'proxy_host'

cluster:
  manager: 172.16.0.30
  manager_docker_port: 2380
  node_docker_port: 2380
  nodes:
    -
      ip: 172.16.0.2
      cpu: 4
      name: 'nursery-node-1'
      group: 'nursery'
    -
      ip: 172.16.0.4
      cpu: 4
      name: 'young-node-1'
      group: 'young'
    -
      ip: 172.16.0.6
      cpu: 8
      name: 'tenured-node-1'
      group: 'tenured'

swarm:
  image: swarm:1.2.0
  strategy: spread
  host_image: docker:1.12.0-rc3-dind
\end{lstlisting}
\end{minipage}

\subsubsection{SSH options}

The CLI is using the protocol \textsc{SSH} to access to the different hosts of your cluster.

\begin{itemize}
  \item \texttt{user} (\emph{mandatory}): the user you want to use to connect to hosts;
  \item \texttt{identity\_file} (\emph{mandatory}): the path to the private key to use to connect to hosts;
  \item \texttt{public\_key\_file} (\emph{optional}): the path to the public key to use to connect to hosts - required only for using CLI's feature for exporting a public key on remote hosts;
  \item \texttt{proxy} (\emph{optional}): the proxy to use to reach the different hosts - required only to use a proxy, it will use the same key than the one for hosts.
\end{itemize}

\subsubsection{Cluster options}

Here are defined all the parameters related to the infrastructure of the cluster:

\begin{itemize}
  \item \texttt{manager} (\emph{mandatory}): IP address of host where the \textsc{Docker Swarm} manager will be deployed;
  \item \texttt{manager\_docker\_port} (\emph{mandatory}): port where the \textsc{Docker Swarm} manager will listen;
  \item \texttt{node\_docker\_port} (\emph{mandatory}): port where the \textsc{Docker Swarm} nodes will listen;
  \item \texttt{nodes} (\emph{mandatory}): the list of the \textsc{Docker Swarm} nodes to set in the cluster, with for each:
  \begin{itemize}
    \item \texttt{ip} (\emph{mandatory}): IP address of the host;
    \item \texttt{cpu} (\emph{mandatory}): number of CPUs of the host;
    \item \texttt{name} (\emph{mandatory}): name to give to the host;
    \item \texttt{group} (\emph{mandatory}): generation which the node is belonging to, either \texttt{nursery}, \texttt{young} or \texttt{tenured}.
  \end{itemize}
\end{itemize}

\subsubsection{Swarm options}

Finally, few parameters about \textsc{Docker Swarm} have to be set:

\begin{itemize}
  \item \texttt{image} (\emph{mandatory}): the image of \texttt{Docker Swarm} to use;
  \item \texttt{strategy} (\emph{mandatory}): the strategy of scheduling to give to \texttt{Docker Swarm} - it will be used only to schedule nodes on the nursery;
  \item \texttt{host\_image} (\emph{mandatory}): the image of \texttt{Docker} to use on the nodes.
\end{itemize}


% -*- root: Document.tex -*-

\subsection{Manage a cluster}
\label{subsec:clustersetup:manage}

We are assuming that your hosts are all reachable by SSH from the CLI.
You can ensure that by running:

\begin{lstlisting}[language=bash, basicstyle=\small]
$ ./remote_swarm_manager.rb ping
\end{lstlisting}

Once the configuration of the hosts to use for creating the cluster has been given, here are the few steps needed to create a cluster:

\begin{itemize}
  \item \emph{Create nodes}: this command will be create the nodes to use in the \textsc{Docker Swarm} cluster, by deploying an image of \textsc{Docker} on the \textsc{Docker} daemon of each host:
    \begin{lstlisting}[language=bash, basicstyle=\small]
  $ ./remote_swarm_manager.rb create-nodes
    \end{lstlisting}
  \item \emph{Create the cluster}: this command will deploy the \textsc{Docker Swarm} cluster:
    \begin{lstlisting}[language=bash, basicstyle=\small]
  $ ./remote_swarm_manager.rb create
    \end{lstlisting}
  \item \emph{Monitor the cluster}: this command will deploy the services responsible of monitoring resources:
    \begin{lstlisting}[language=bash, basicstyle=\small]
  $ ./remote_swarm_manager.rb monitor
    \end{lstlisting}
\end{itemize}

All these commands can be run at once:

\begin{lstlisting}[language=bash, basicstyle=\small]
    $ ./remote_swarm_manager.rb create-nodes create monitor
\end{lstlisting}

If you want to recreate a cluster, use the command \texttt{remove} before, \textit{e.g.}:

\begin{lstlisting}[language=bash, basicstyle=\small]
    $ ./remote_swarm_manager.rb remove create-nodes create monitor
\end{lstlisting}



\newpage

% -*- root: Document.tex -*-

\section{Genpack Setup}
\label{sec:genpacksetup}

\GP{} can be downloaded at \genpackschedulerrepo{}.

\GP{} is requiring a well configured \textsc{Docker Swarm} cluster and the monitoring system deployed on the cluster (cf. Subsection~\ref{subsec:clustersetup:manage}).



% -*- root: Document.tex -*-

\subsection{Requirements and dependencies}
\label{subsec:genpacksetup:requirements}

Requirements and the installation of dependencies for \textsc{Genpack} are exactly like those described in Subsection~\ref{subsec:clustersetup:requirements}.


% -*- root: Document.tex -*-

\subsection{Configuration}
\label{subsec:genpacksetup:config}

\subsubsection{Global Configuration}

The global configuration has to be given into the YAML file \texttt{config.yml} at the root of \GP{}'s code.
This file has to be build like in the example~\ref{genpack-scheduler-config}.
All configurations options are detailed below.

\begin{minipage}{\linewidth} %avoid splitting
% \vspace{10pt}
\begin{lstlisting}[language=YAML,caption={Configuration file example for \GP{} scheduler.},label=genpack-scheduler-config][t]
genpack:
  docker_url: "tcp://localhost:2375"
  swarm_url: "tcp://localhost:2380"
  redis_host: 'localhost'
  redis_port: 6379
  redis_docker_image: 'redis:3.2.1-alpine'
  generations:
    - nursery
    - young
    - tenured
\end{lstlisting}
\end{minipage}

\begin{itemize}
  \item \texttt{docker\_url} (\emph{optional}): the url to reach the \textsc{Docker} daemon on manger host;
  \item \texttt{swarm\_url} (\emph{optional}): the url to reach the \textsc{Docker Swarm} manager;
  \item \texttt{redis\_host} (\emph{optional}): the address of the \textsc{Redis} server;
  \item \texttt{redis\_docker\_image} (\emph{optional}): the image to use for deploying the \textsc{Redis} server;
  \item \texttt{swarm\_url} (\emph{optional}): the url to reach the \textsc{Docker Swarm} manager;
  \item \texttt{generations} (\emph{mandatory}): the list of the different generations - its order is important because it will define the order between generations.
\end{itemize}

All mandatory values above are set by default to the ones given in the example~\ref{genpack-scheduler-config}.

\subsubsection{Generation Cycles Configuration}

The cycles of migration between the generations are executed like some tasks.
These tasks are some jobs performed by \textsc{Resque}\footnote{\textsc{Resque} is a \textsc{Redis}-backed \textsc{Ruby} library for creating background jobs, placing them on multiple queues, and processing them later: \url{https://github.com/resque/resque}.} workers and launched by a task scheduler, \textsc{Resque Scheduler}\footnote{\textsc{Resque Scheduler}, light-weight job scheduling system built on top of \textsc{Resque}: \url{https://github.com/resque/resque-scheduler}.}.
The frequency of generation cycles is configured in the file \texttt{config/resque\_schedule.yml} (cf. Listing~\ref{resque-scheduler-config}), using the \textsc{Crontab} syntax\footnote{\textsc{Crontab} Quick Reference: \url{http://www.adminschoice.com/crontab-quick-reference}.}.
In the example given here, each \emph{GenerationCycle} from \texttt{nursery} to \texttt{young} generations is executed every 5 minutes, each one from \texttt{young} to \texttt{tenured} generations each 10 minutes.

\begin{minipage}{\linewidth} %avoid splitting
% \vspace{10pt}
\begin{lstlisting}[language=YAML,caption={Configuration file for \textsc{Resque Scheduler}.},label=resque-scheduler-config][t]
  generation_nursery_cycle:
    cron: "*/5 * * * *"
    class: GenerationCycle
    description: "Run a cycle on generation nursery"
    args: nursery

  generation_young_cycle:
    cron: "*/10 * * * *"
    class: GenerationCycle
    description: "Run a cycle on generation young"
    args: young
\end{lstlisting}
\end{minipage}


% -*- root: Document.tex -*-

\subsection{Run \GP{}}
\label{subsec:genpacksetup:run}

We are assuming that your \textsc{Docker Swarm} manager, the \textsc{Docker} daemon from your manager host, the servers \textsc{Redis} and \textsc{InfluxDB} are all reachable from where you are launching \GP{}.
Typically, \GP{} is launched on the \textsc{Docker Swarm} manager host, but you can also launch it remotely if each service is reachable from thre (you can eventually open some SSH tunnels for that).

Once all the requirements described above are checked, run \GP{} from the root of its code directory with the command:

\begin{lstlisting}[basicstyle=\small]
$ bundle exec rake genpack:init
\end{lstlisting}

Then you can follow \GP{}'s logs from the file \texttt{log/genpack.log}.

You can now run some containers when it is explained at the beginning of this chapter~\ref{chap:usermanual}.


