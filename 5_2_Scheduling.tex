% -*- root: Document.tex -*-

\section{Container Scheduling}
\label{sec:scheduling}

Once the container profiles are identified and the associated resource envelopes have been computed by the monitoring module of \GP{}, the scheduling module builds on these resources estimations to identify the best fitting node for each of the container executing in the \emph{nursery} generation.

\begin{algorithm}[ht!]
  \begin{algorithmic}[1]
    \Procedure{Schedule}{$envelopes$, $nodes$}
        \State $containers \gets \Call{Blend}{envelopes}$\label{alg:scheduling:blend}
        \For{$c \in containers$}
            \State $res_c \gets \Call{Resources}{$c$}$
            \For{$n \in nodes$}\Comment{Find the best node for $c$}\label{alg:scheduling:find-best-node}
                \State $avail_n \gets$ \Call{Availability}{$n$}
                \If{\Call{Matches}{$res_c$, $avail_n$}}\label{alg:scheduling:pick-best-node}%\Comment{$cont$ fits in $node$}
                    \State $n \gets$ \Call{Update}{$n$, $res_c$}\label{alg:scheduling:update-node}
                    \State $nodes \gets$ \Call{ShiftLeft}{$nodes$, $n$}\label{alg:scheduling:shift}
                    \State \Call{Migrate}{$c$, $n$}\Comment{Async. migration}\label{alg:scheduling:migrate}
                    \State \textbf{break}\Comment{$c$ succeeds to be scheduled}
                \EndIf
            \EndFor
            \State \Call{Escape}{$c$}\Comment{$c$ fails to be scheduled}\label{alg:scheduling:escape}
        \EndFor
    \EndProcedure
    \vspace{5pt}
    \Function{Blend}{$envelopes$}\label{alg:scheduling:blend-begin}
        \State $list \gets \{\}$
        \State $emptied \gets \textbf{true}$
        \Repeat\Comment{Blend until all envelopes are emptied}
            \State $emptied \gets \textbf{true}$
            \For{$env \in envelopes$}
                \If{\textbf{not} \Call{IsEmpty}{$env$}}
                    \State $list \gets list~\Vert~\Call{Head}{env}$ %\Comment{appends}
                    \State $env \gets \Call{Tail}{env}$ %\Comment{appends}
                    \State $emptied \gets \Call{IsEmpty}{$env$}$
                \EndIf
            \EndFor
        \Until{$emptied$}
        \State \Return $list$
    \EndFunction\label{alg:scheduling:blend-end}
    \vspace{5pt}
    \Function{ShiftLeft}{$nodes$, $node$}\label{alg:scheduling:shift-begin}
        \State $i \gets \Call{Index}{nodes, node}$
        \State $n \gets \Call{Length}{nodes}$
        \State $list \gets nodes[0:i-1]~\Vert~nodes[i+1:n-1]$
        \State $i \gets 0$
        \State $score \gets \Call{Availability}{node}$
        \While{$score \ge \Call{Availability}{list[i]}$}
            \State $i \gets i+1$
        \EndWhile
        \State \Return $list[0:i-1]~\Vert~node~\Vert~list[i:n-1]$
    \EndFunction\label{alg:scheduling:shift-end}
  \end{algorithmic}
  \caption{Container scheduling in \GP{}.}\label{alg:scheduling}
\end{algorithm}

More specifically, Algorithm~\ref{alg:scheduling} describes the scheduling strategy applied by \GP{} to migrate a set of profiled containers at runtime.
The scheduling phase is triggered for a given set of container $envelopes$ and available $nodes$.
The algorithm starts by homogeneously blending the content (\emph{i.e.}, container descriptions) of the envelopes (line~\ref{alg:scheduling:blend} and lines~\ref{alg:scheduling:blend-begin}--\ref{alg:scheduling:blend-end}) to increase of the diversity of containers per node.
From there, it iterates over this ordered set of $containers$ to be scheduled (line~\ref{alg:scheduling:find-best-node}) and picks the first node $n$ among the ordered list of available $nodes$ (as explained in the previous section) that matches the resources requirements of the container $c$ (line~\ref{alg:scheduling:pick-best-node}).
Upon resource matching, the estimation of the node's resource availability is updated accordingly (line~\ref{alg:scheduling:update-node}) and the order of available $nodes$ is refreshed by shifting the selected node $n$ towards the head of the ordered list (line~\ref{alg:scheduling:shift} and lines~\ref{alg:scheduling:shift-begin}--\ref{alg:scheduling:shift-end}).
By reasoning on such resources estimations, \GP{} can therefore trigger the migration of the container $c$ to the node $n$ asynchronously (line~\ref{alg:scheduling:migrate}) and thus keep scheduling the remaining nodes in parallel.
If none of the available $nodes$ fits the resource requirements of the container $c$, the escape trigger of \GP{} (line~\ref{alg:scheduling:escape}) is used to provision a new node within the \emph{young} generation, migrate the container $c$ on this new node, and add the node to the list of available nodes.

The intuition behind this algorithm is to ensure a better distribution of resource consumption and power efficiency of the infrastructure by increasing the entropy (in term of resource diversity) of the containers deployed within each node.
Furthermore, by reasoning on resource estimations (computed during the monitoring phase) instead of real-time metrics, \GP{} can increase the scheduling parallelism and thus absorb the delay induced by the container migration process (including state snapshotting, binary transfer, remote provisioning steps).
