% -*- root: Document.tex -*-

\subsection{Manage a cluster}
\label{subsec:clustersetup:manage}

We are assuming that your hosts are all reachable by SSH from the CLI.
You can ensure that by running:

\begin{lstlisting}[language=bash, basicstyle=\small]
$ ./remote_swarm_manager.rb ping
\end{lstlisting}

Once the configuration of the hosts to use for creating the cluster has been given, here are the few steps needed to create a cluster:

\begin{itemize}
  \item \emph{Create nodes}: this command will be create the nodes to use in the \textsc{Docker Swarm} cluster, by deploying an image of \textsc{Docker} on the \textsc{Docker} daemon of each host:
    \begin{lstlisting}[language=bash, basicstyle=\small]
  $ ./remote_swarm_manager.rb create-nodes
    \end{lstlisting}
  \item \emph{Create the cluster}: this command will deploy the \textsc{Docker Swarm} cluster:
    \begin{lstlisting}[language=bash, basicstyle=\small]
  $ ./remote_swarm_manager.rb create
    \end{lstlisting}
  \item \emph{Monitor the cluster}: this command will deploy the services responsible of monitoring resources:
    \begin{lstlisting}[language=bash, basicstyle=\small]
  $ ./remote_swarm_manager.rb monitor
    \end{lstlisting}
\end{itemize}

All these commands can be run at once:

\begin{lstlisting}[language=bash, basicstyle=\small]
    $ ./remote_swarm_manager.rb create-nodes create monitor
\end{lstlisting}

If you want to recreate a cluster, use the command \texttt{remove} before, \textit{e.g.}:

\begin{lstlisting}[language=bash, basicstyle=\small]
    $ ./remote_swarm_manager.rb remove create-nodes create monitor
\end{lstlisting}
