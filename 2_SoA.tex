% -*- root: Document.tex -*-

\chapter{State-of-the-Art}
\label{chap:soa}

Resource management and scheduling is an important topic.
Many researchers have addressed various aspects of scheduling resources during the last decades.
Scheduling has been addressed in the context of GRID computing~\cite{buyya2000nimrod}, distributed systems~\cite{tannenbaum2001condor}, HPC~\cite{Jackson2001}, batch processing~\cite{capit2005batch}, MapReduce~\cite{Yarn}, and more recently in the context of VM~\cite{litvinski2013openstack} and container scheduling~\cite{Burns:2016:BOK:2930840.2890784} in large clusters.

Distributed job schedulers like the \textsc{Condor} scheduler~\cite{tannenbaum2001condor} performs a \emph{match making} between a job waiting to run and the machines available to run jobs.
Hence, each job explicitly describes its resource requirements and also a \emph{rank expression} that permits the scheduler to select the machine that is most suited to run this job.
Also, the resources of a machine have to be explicitly described.
In \GP, we avoid the need to describe jobs and machines by performing an automatic profiling of the containers and nodes (cf. Section~\ref{sec:monitoring}).

The \textsc{OpenStack Nova} scheduler does not consider CPU load for the assignment of VMs~\cite{litvinski2013openstack}.
The scheduling in \textsc{OpenStack}, no matter the selected strategy, is rather based on statically defined RAM and CPU size of the VM, known as flavors~\cite{litvinski2013openstack}.
In our experience, the simple round-robin scheduler results in many cases in situations where all hosts run some VMs and none of the hosts can be switched off to reduce the energy consumption (cf. Chapter~\ref{chap:motivation}).

\textsc{OptSched}\cite{knauth2012energy} compares the energy implications of a \emph{round robin} scheduler, a \emph{first fit} scheduler, and an \emph{optimized} scheduler that knows the run time of (some of) the VMs upon scheduling.
Knowing the run times before starting a VM helps reduce the total energy consumed by a cluster.
In \GP, however, run times are not known \emph{a priori} and \GP is able to automatically learn the profile that is used by the scheduler along generations to improve the energy efficiency of the cluster (cf. Section~\ref{sec:evaluation}).

\textsc{Yarn}~\cite{Yarn} is a two-level scheduler that can handle multiple workloads on the same cluster.
It is request-based and supports locality of scheduling decisions such that jobs can, for example, access data on local disks to avoid remote accesses via the network.
Nonetheless, the scheduling in \textsc{Yarn} implements a strategy close to the \emph{spread} strategy of \textsc{Docker Swarm}, thus suffering from the same limitations in terms of power usage efficiency.

Google developed a series of container management systems during the last 10 years~\cite{Burns:2016:BOK:2930840.2890784}: \textsc{Borg}, \textsc{Omega}, and more recently \textsc{Kubernetes}.
Initially, Google started with a centralized container management system called \textsc{Borg}, which remains the main system in use by Google~\cite{DBLP:conf/eurosys/VermaPKOTW15}.
\textsc{Omega} is based on the lessons learned from \textsc{Borg} and has a principled architecture that includes a centralized transactional store and an optimistic concurrency control.
In particular, the \textsc{Omega} architecture supports multiple concurrent schedulers.
Finally, \textsc{Kubernetes} is an open source container system that focuses on simplifying the task of application developers and has less focus on maximizing the utilization of clusters---which is the focus of \textsc{Omega} and \textsc{Borg}.
Compared to \GP{}, all these approaches does not incorporate the concept of generations within the cluster to automatically learn about the container profiles at runtime.

\textsc{Docker Swarm} is very similar to \textsc{Kubernetes} in that it aims to support \emph{cloud native} applications.
\textsc{Swarm} permits users to define applications consisting of a set of containers.
The focus is on simplifying the typical tasks of the application developers like load balancing, elasticity, and high availability.
Unlike \GP, the main goal of \text{Swarm} is not on ensuring a high utilization of a compute cluster, but this document demonstrates how we succeed to extend it in order to address this concern.
