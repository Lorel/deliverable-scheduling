% -*- root: Document.tex -*-

\subsection{Configuration}
\label{subsec:clustersetup:config}

The configuration has to be given into the YAML file \texttt{config.yml} at the root of tool's code.
This file has to be build like in the example~\ref{remote-swarm-manager-config}.
All configurations options are detailed below.

\begin{minipage}{\linewidth} %avoid splitting
% \vspace{10pt}
\begin{lstlisting}[language=YAML,caption={Configuration file example for \textsc{RemoteSwarmManager} CLI.},label=remote-swarm-manager-config][t]
ssh:
  user: 'host_user'
  identity_file: 'path/to/ssh_keys/id_rsa'
  public_key_file: 'path/to/ssh_keys/id_rsa.pub'
  proxy:
    user: 'proxy_user'
    host: 'proxy_host'

cluster:
  manager: 172.16.0.30
  manager_docker_port: 2380
  node_docker_port: 2380
  nodes:
    -
      ip: 172.16.0.2
      cpu: 4
      name: 'nursery-node-1'
      group: 'nursery'
    -
      ip: 172.16.0.4
      cpu: 4
      name: 'young-node-1'
      group: 'young'
    -
      ip: 172.16.0.6
      cpu: 8
      name: 'tenured-node-1'
      group: 'tenured'

swarm:
  image: swarm:1.2.0
  strategy: spread
  host_image: docker:1.12.0-rc3-dind
\end{lstlisting}
\end{minipage}

\subsubsection{SSH options}

The CLI is using the protocol \textsc{SSH} to access to the different hosts of your cluster.

\begin{itemize}
  \item \texttt{user} (\emph{mandatory}): the user you want to use to connect to hosts;
  \item \texttt{identity\_file} (\emph{mandatory}): the path to the private key to use to connect to hosts;
  \item \texttt{public\_key\_file} (\emph{optional}): the path to the public key to use to connect to hosts - required only for using CLI's feature for exporting a public key on remote hosts;
  \item \texttt{proxy} (\emph{optional}): the proxy to use to reach the different hosts - required only to use a proxy, it will use the same key than the one for hosts.
\end{itemize}

\subsubsection{Cluster options}

Here are defined all the parameters related to the infrastructure of the cluster:

\begin{itemize}
  \item \texttt{manager} (\emph{mandatory}): IP address of host where the \textsc{Docker Swarm} manager will be deployed;
  \item \texttt{manager\_docker\_port} (\emph{mandatory}): port where the \textsc{Docker Swarm} manager will listen;
  \item \texttt{node\_docker\_port} (\emph{mandatory}): port where the \textsc{Docker Swarm} nodes will listen;
  \item \texttt{nodes} (\emph{mandatory}): the list of the \textsc{Docker Swarm} nodes to set in the cluster, with for each:
  \begin{itemize}
    \item \texttt{ip} (\emph{mandatory}): IP address of the host;
    \item \texttt{cpu} (\emph{mandatory}): number of CPUs of the host;
    \item \texttt{name} (\emph{mandatory}): name to give to the host;
    \item \texttt{group} (\emph{mandatory}): generation which the node is belonging to, either \texttt{nursery}, \texttt{young} or \texttt{tenured}.
  \end{itemize}
\end{itemize}

\subsubsection{Swarm options}

Finally, few parameters about \textsc{Docker Swarm} have to be set:

\begin{itemize}
  \item \texttt{image} (\emph{mandatory}): the image of \texttt{Docker Swarm} to use;
  \item \texttt{strategy} (\emph{mandatory}): the strategy of scheduling to give to \texttt{Docker Swarm} - it will be used only to schedule nodes on the nursery;
  \item \texttt{host\_image} (\emph{mandatory}): the image of \texttt{Docker} to use on the nodes.
\end{itemize}
